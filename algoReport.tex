\documentclass{article}
\usepackage[T1]{fontenc}
\usepackage{listings}
\usepackage[english, greek]{babel}
\usepackage{tikz}
\usepackage[intlimits]{amsmath}
\usepackage{commath}
\usepackage{amssymb}
\usepackage{graphicx}
\setlength{\parindent}{0pt}


\lstset{frame=tb,
  language=C,
  aboveskip=3mm,
  belowskip=3mm,
  showstringspaces=false,
  columns=flexible,
  basicstyle={\small\ttfamily},
  numbers=none,
  numberstyle=\tiny\color{gray},
  stringstyle=\color{mauve},
  breaklines=true,
  breakatwhitespace=true,
  tabsize=3
}

\begin{document}

\begin{titlepage}
   \vspace*{\stretch{1.0}}
   \begin{center}
      \selectlanguage{english}
   	  \Large\textbf{Algorithms Spring 2024}\\
   	  \selectlanguage{greek}
      \Large\textbf{\selectlanguage{english}Project Report\selectlanguage{greek}}\\
      \large\textit{Δημήτρης Παπαδημτρίου-Βασίλης Μαδέσης}\\
      \large\textit{AEM: 03750-03748}
   \end{center}
   \vspace*{\stretch{2.0}}
\end{titlepage}

Εύρεση \selectlanguage{english}CPL\selectlanguage{greek}:\bigbreak

Για να βρούμε το \selectlanguage{english}CPL\selectlanguage{greek} ενός γράφου χρησημοποιήσαμε\selectlanguage{english} Breadth First Search\selectlanguage{greek} ώστε να υπολογίσουμε όλα τα μήκοι των σύντομων μονοπάτιων μεταξύ όλων των κόμβων.Συγκεκριμένα υλοποιούμε μια συνάρτηση η οποία λαμβάνει ως \selectlanguage{english}input\selectlanguage{greek} το \selectlanguage{english}source node\selectlanguage{greek} και έχει ένα πίνακα που κρατάει τις αποστάσεις όλων των κόμβων από τον κόμβο \selectlanguage{english}source\selectlanguage{greek} μαζί με ένα πίνακα με τους γονείς κάθε κόμβου.Ο πίνακας γονιών θα χρησιποιηθεί αργότερα για το Β ερώτημα όπου θα εξηγήσουμε την χρήση του.Τελός η συνάρτηση αυτή επιστρέφει τους πίνακες αύτους ενώ αν της δωθεί συγκεκριμένος \selectlanguage{english}destination\selectlanguage{greek} κομβός τότε θα τρέχει μέχρι να βρεί το κόμβο αυτό επιστρέφοντας επιτυχία με 1 και αποτυχία με 0.\bigbreak

Για την υλοποίηση της \selectlanguage{english} Breadth First Search\selectlanguage{greek} χρήσιμοποιήουμε μια \selectlanguage{english}queue (FIFO \selectlanguage{greek}ούρα) ώστε να τοποθετούμε μέσα της τους κόμβους που πρέπει να επισκεφτούμε μετά από μια επίσκεψη κόμβου σε κάθε επανάληψη.\bigbreak

Η συνάρτηση λειτουργεί ως εξής:

\selectlanguage{english}
\begin{lstlisting}
pathSearch()
distance[size of graph] -> 0

Insert in queue node S
while(Queue is not empty)
	parent node <- remove from queue the first element
	
	for all child nodes of parent node
		if(distance[child node] is 0 && child node != S)
			Put child node into queue
			distance[child node] = distance[parent node] + 1
			parent[child node] = parent node
		else if(distance[child node] == distance[parent node] + 1)
			parent[child node] -> add parent to the list of parents for child node
		if (distance[child node] == distance[parent node] + 1 && child node == destination node)
			return 1
	end	
return 0
end
\end{lstlisting}\bigbreak
\selectlanguage{greek}

Για να βρούμε το \selectlanguage{english}CPL\selectlanguage{greek} εκτελούμε κατα επανάληψη την προηγούμενη συνάρτηση για όλους τους κόμβους και μηδενίζουμε τις επαναλαμβανόμενες αποστάσεις που προκύπτουν στον πίνακα αποστάσεων που μας επιστρέφει η συνάρτηση και έπειτα προσθέτουμε στο συνολικό άθροισμα, το άθροισμα των \selectlanguage{english}CPL\selectlanguage{greek} του πίνακα αυτού.\pagebreak

\selectlanguage{english}
\begin{lstlisting}
Sum of all cpls = 0;

for all nodes inside the graph
	distances array = execute pathSearch
	distances array -> distance array with zeroed the recurring cpls
	Sum of all cpls = distances array + Sum of all cpls
end

cpl = Sum of all cpls / binomial coefficient(number of nodes,2)
\end{lstlisting}\bigbreak
\selectlanguage{greek}

Για να υπολογίσουμε τον συνολικό αριθμό των συντομοτέρων μονοπατιών \selectlanguage{english}(SPs)\selectlanguage{greek} μεταξύ κάθε δυνατού ζεύγους κόμβων $j$ και $k$ στα οποία μεσολαβεί κάθε ακμή $e$ του γραφήματος.Θα χρησιμοποιήσουμε τον πίνακα γονέων που μας επιστρέφει η συνάρτηση
\selectlanguage{english}pathSearch()\selectlanguage{greek}.\bigbreak

Ο πίνακας γονέων είναι ένας πίνακας που σε κάθε θέση $i$ έχει τον γονέα του κόμβου $i+1$.Γονέας ενός κόμβου είναι ο γειτονικός κόμβος από τον οποίο διαβάστηκε και μπήκε στον πίνακα αποστάσεων ο κόμβος παιδί(στην περιπτωσή μας ο κόμβος $i+1$).Σε περίπτωση ύπαρξης δύο ή περισσότερων σύντομων μονοπατιών μεταξύ δύο κόμβων ο κόμβος \selectlanguage{english}destination\selectlanguage{greek} θα έχει δύο γονείς καθώς και οι δύο είναι μέρος σύντομου μονοπατιού.\bigbreak

Οι συμλήρωση του πίνακα γονέων φαίνεται στο παρακάτω \selectlanguage{english}code snippet\selectlanguage{greek} της συνάρτησης\selectlanguage{english} pathSearch\selectlanguage{greek}:

\selectlanguage{english}
\begin{lstlisting}
if(distance[child node] is 0 && child node != S)
			Put child node into queue
			distance[child node] = distance[parent node] + 1
			parent[child node] = parent node
else if(distance[child node] == distance[parent node] + 1)
			parent[child node] -> add parent to the list of parents for child node
\end{lstlisting}\bigbreak
\selectlanguage{greek}
\bigbreak
Αν ο κόμβος έχει παραπάνω από έναν γονέα τότε δημιουργείται μια λίστα με του γονείς του κόμβου αυτού στην αντίστοιχη θέση του πίνακα.\bigbreak

Προσθέτοντας στον προήγουμενο κώδικα, για κάθε επανάληψη της 'λούπας' παίρνουμε τον πίνακα γονέων και για κάθε ζευγάρι κόμβων\selectlanguage{english} source-destination\selectlanguage{greek} διατρέχουμε το πίνακα ξεκίνοντας από το \selectlanguage{english}destination\selectlanguage{greek} και ακολουθώντας τον γονέα του και από εκεί ακολουθώντας τον γονέα του γονέα του ,φτάνουμε στο \selectlanguage{english}source\selectlanguage{greek} καταγράφοντας τα \selectlanguage{english}edges\selectlanguage{greek} που συμμετείχαν στο δρόμο μέχρι το \selectlanguage{english}source\selectlanguage{greek}.\pagebreak

Το \selectlanguage{english}path\selectlanguage{greek} για κάθε ζεύγος καθώς και τα \selectlanguage{english}edges\selectlanguage{greek} που συμμετέχουν σε αυτό βρίσκεται από τον ψευδοκώδικα:

\selectlanguage{english}
\begin{lstlisting}
countEdgesPath()
	current node = destination node
	while(parent of current node != Source)
	  array of edges [parent node][current node] ++
	  array of edges [parent node][current node] ->add info about the edge
	end
	
	if (there are more than one parents for destination node)
		repeat the process for the other parent of destination node
end
\end{lstlisting}
\selectlanguage{greek}

Ο συνολικός ψευδοκώδικας προκύπτει:
\bigbreak
\selectlanguage{english}

\begin{lstlisting}
Sum of all cpls = 0;

for all nodes inside the graph
	distances array = execute pathSearch
	distances array -> distance array with zeroed the recurring cpls
	Sum of all cpls = distances array + Sum of all cpls
	
	for all combinations of sources and destination nodes inside a graph
		countEdgesPath() fill the edge array
	end 
end

cpl = Sum of all cpls / binomial coefficient(number of nodes,2)
\end{lstlisting}\bigbreak
\selectlanguage{greek}

Τέλος έξω απο την "λούπα" κάνουμε \selectlanguage{english}merge sort\selectlanguage{greek} το πίνακα \selectlanguage{english}edge array\selectlanguage{greek} και επιστρέφουμαι το πρώτο \selectlanguage{english}edge\selectlanguage{greek} του πίνακα το οποίο και θα αφαιρέσουμε.Το δύο προηγούμενα ερωτήματα τα εκτελεί η συνάρτηση \selectlanguage{english}cpl sp\selectlanguage{greek}.\pagebreak

\selectlanguage{english}
\begin{lstlisting}
cpl_sp()
Sum of all cpls = 0;

for all nodes inside the graph
	distances array = execute pathSearch
	distances array -> distance array with zeroed the recurring cpls
	Sum of all cpls = distances array + Sum of all cpls
	
	for all combinations th source and destination nodes inside the graph
		countEdgesPath() fill the edge array
	end 
end

cpl = Sum of all cpls / binomial coefficient(number of nodes,2)
mergesort(edge array)

return top elemnent in edge array
end
\end{lstlisting}\bigbreak
\selectlanguage{greek}

Συνεπώς το \selectlanguage{english}complexity\selectlanguage{greek} της συνάρτησης προκύπτει ως εξής:\bigbreak

To \selectlanguage{english}complexity\selectlanguage{greek} της αναζήτησης είναι $Ο(V+E)$ όπου $V$ κόμβοι και $Ε$ ακμές.\bigbreak

Για την συνάρτηση μηδενισμού έχουμε \selectlanguage{english}complexity\selectlanguage{greek}
$O(V)$ όπως το ίδιο και για την συνάστηση αθροίσματος.\bigbreak

Η συνάστηση επαναληπτική εκτέλεση της συνάστησης\selectlanguage{english}countEdgesPath()\selectlanguage{greek} έχει \selectlanguage{english}comlexity\selectlanguage{greek}
$O(V*Biggest Path Length)$ καθώς εκτελείται περίπου $Ο(V)$ φορές για ένα δεδομένο πίνακα γονέων και η διάτρεξη του πίνακα γονέων έχει στην μεγαλύτερη περίπτωση μεταξύ όλων των $Ο(Biggest Path Length)$ πράξεις.\bigbreak

Άρα το συνολικό \selectlanguage{english}complexity\selectlanguage{greek} εφόσον γίνονται $V$ επαναλήψης είναι 

$Ο(V) * [O(V+E)+O(V)+O(V)+Ο(V*Biggest Path Length)] \ = \ O(V^2 + VE)$\bigbreak

Eφόσον έχουμε εκτελέσει την συνάρτηση \selectlanguage{english}cpl sp\selectlanguage{greek} τότε κάνουμε αφαίρεση της ακμής μεταξύ των δύο κόμβων αφαιρόντας τις μεταξύ ακμές από τις αντίστοιχες λίστες γειτνίασης τους.\bigbreak

Έπειτα για να ελένξουμε για συνοχή του γράφου εκτελούμε την \selectlanguage{english}pathSearch\selectlanguage{greek} με στόχο να βρούμε το κόμβο με τον οποίο κόπηκε η απευθείας σύνδεση.
Εάν τον βρούμε θα επιστραφεί επιτυχία από την συνάρτηση και θα σημαίνει οτι ο γράφος παραμένει συνεκτικός αλλίως σε περίπτωση αποτυχίας θα σημαίνει οτι ο γράφος χωρίστηκε.
\end{document}